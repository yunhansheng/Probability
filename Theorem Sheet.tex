\documentclass[hidelinks,11pt]{article}

\usepackage[margin=1in]{geometry}
\usepackage{amsmath,amsthm,amsfonts}
\usepackage[utf8]{inputenc}
\usepackage{amssymb}
\usepackage[mathscr]{eucal}
\usepackage{graphicx}
\usepackage{listings}
\usepackage{xcolor}
\usepackage[OT1]{fontenc}
\usepackage{physics}
\usepackage{tikz-cd}
\usepackage{xpatch}
\usepackage{nicefrac}
\usepackage{mathtools}
\usepackage{environ}

\setlength\parindent{0pt}

\newtheoremstyle{dotless}{}{}{\itshape}{}{\bfseries}{}{ }{}

\theoremstyle{definition}
\newtheorem*{defin}{DEF}

\theoremstyle{dotless}
\newtheorem{prop}{PROP}[section]
\newtheorem*{corollary}{Corollary}

\theoremstyle{remark}
\newtheorem*{remark}{Remark}

\usepackage{hyperref}
\hypersetup{colorlinks=false}

\DeclareMathOperator{\Var}{Var}
\DeclareMathOperator{\E}{\mathbb{E}}
\DeclareMathOperator{\R}{\mathbb{R}}

\let\oldproof\proof
\renewcommand{\proof}{\color{blue}\oldproof}

%\NewEnviron{killcontents}{}\let\proof\killcontents\let\endproof\endkillcontents



\begin{document}

\section{Measure Theory}

\section{Basic Proabbility}

\subsection{Basic notions}

\begin{prop}\label{Prop 2.1}
If $F$ is the distribution function of random variable $X$, then $F$ is:\begin{itemize}
    \item non-decreasing,
    \item right-continuous: $\lim_{\Delta t\to0}F(\Delta t+t)=F(t)$,
    \item $\lim_{t\to-\infty}F(t)=0$, and $\lim_{t\to\infty}F(t)=1$.
\end{itemize}
Moreover, $F$ determines the law $\mu_X$ of $X$ uniquely. Conversely if $F:\mathbb{R}\to[0,1]$ satisfies the listed properties, then there is a probability space $(\Omega, \mathcal{F},\mathbb{P})$ and a random variable $X$ such that $F$ is the distribution function of $X$.
\end{prop}

\begin{prop}\label{Prop 2.2}
Let $X$ and $Y$ be random variables on measure space $(\Omega,\mathcal{F},\mathbb{P})$. Let $p,q\in[1,\infty]$ be such that $pq=p+q$.
\begin{itemize}
    \item \textup{\textbf{Jensen's Inequality}} If $X\in L^1(\mathbb{P})$ and $\phi:I\to\mathbb{R}$ is convex on some $I\subset\mathbb{R}$: for all $x,y\in I$ and $t\in[0,1]$,
    \[\phi(tx+(1-t)y)\leq t\phi(x)+(1-t)\phi(y).\]
    Then
    \[\phi(\E[X])\leq\E[\phi(X)].\]
    \item \textup{\textbf{Hölder's Inequality}} If $X\in L^p(\mathbb{P})$ and $Y\in L^q(\mathbb{P})$, then
    \[\norm{XY}_1\leq\norm{X}_p\norm{Y}_q.\]
    \item \textup{\textbf{Minkowski's Inequality}} If $X\in L^p(\mathbb{P})$ and $Y\in L^q(\mathbb{P})$, then
    \[\norm{X+Y}_{p}\leq\norm{X}_p\norm{Y}_p.\]
\end{itemize}
\end{prop}

\begin{prop}\label{Prop 2.2}
Let $X$ be a random variable on measure space $(\Omega,\mathcal{F},\mathbb{P})$.
\begin{itemize}
    \item \textup{\textbf{Markov's Inequality}} If $X$ is non-negative and $t\geq0$, then
    \[t\mathbb{P}(X\geq t)\leq\mathbb{E}(X).\]
    \item \textup{\textbf{Chebyshev's Inequality}} For any $t\geq0$,
    \[t^2\mathbb{P}(\abs{X-\mathbb{E}(X)}\geq t)\leq\Var(X).\]
    \item \textup{\textbf{Chernoff bound}} Given $m_X(\lambda)=\mathbb{E}(e^{\lambda X})$ the moment-generating function of $X$ and $\lambda>0$,
    \[e^{\lambda t}\mathbb{P}(X\geq t)\leq m_X(\lambda).\]
\end{itemize}
\end{prop}

\begin{prop}
covariance
\end{prop}

\subsection{Independence and tail events}

\subsection{Convergence of random variables}

\begin{prop}
\[\begin{tikzcd}
&L^p\arrow[dr,Rightarrow]\\
&&\mathbb{P}\arrow[r,Rightarrow]& d\arrow[r,Leftrightarrow]&\E[f(X_n)]\to\E[f(X)]\ \forall\textrm{ bounded continuous }f:\mathbb{R}\to\mathbb{R}\\
&a.s.\arrow[ur,Rightarrow]
\end{tikzcd}\]
\[\begin{tikzcd}
L^p \arrow[rd, Rightarrow] \arrow[dd, "subsequence" description, bend left]       &                                                                                                                      &                                                            &    \\
                                                                                  & \mathbb{P} \arrow[r, Rightarrow, bend right] \arrow[lu, "\textrm{uniform bounded}"', bend right] \arrow[ld, "subsequence", bend left] & d \arrow[r, Leftrightarrow] \arrow[l, "X_n\xrightarrow{d} C"', bend right] & \begin{matrix}
                    \E[f(X_n)]=\E[f(X)]\\              \forall\textrm{ bounded continuous }f:\mathbb{R}\to\mathbb{R} \end{matrix}\\
a.s. \arrow[ru, Rightarrow] \arrow[uu, "\textrm{uniform integrable (to }L^1\textrm{ only)}", bend left=49] &                                                                                                                      &                                                            & 
\end{tikzcd}\]
\end{prop}


\subsection{Laws of large numbers}

\subsection{Tightness and characteristic functions}



\end{document}