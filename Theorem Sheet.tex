\documentclass[hidelinks,11pt]{article}

\usepackage[margin=1in]{geometry}
\usepackage{amsmath,amsthm,amsfonts}
\usepackage[utf8]{inputenc}
\usepackage{amssymb}
\usepackage[mathscr]{eucal}
\usepackage{graphicx}
\usepackage{listings}
\usepackage{xcolor}
\usepackage[OT1]{fontenc}
\usepackage{physics}
\usepackage{tikz-cd}
\usepackage{xpatch}
\usepackage{nicefrac}
\usepackage{mathtools}
\usepackage{environ}

\setlength\parindent{0pt}

\newtheoremstyle{dotless}{}{}{\itshape}{}{\bfseries}{}{ }{}

\theoremstyle{definition}
\newtheorem*{defin}{DEF}

\theoremstyle{dotless}
\newtheorem{prop}{PROP}[section]
\newtheorem*{corollary}{Corollary}

\theoremstyle{remark}
\newtheorem*{remark}{Remark}

\usepackage{hyperref}
\hypersetup{colorlinks=true,linkcolor=magenta}

\DeclareMathOperator{\Var}{Var}
\DeclareMathOperator{\Cov}{Cov}
\DeclareMathOperator{\E}{\mathbb{E}}
\DeclareMathOperator{\R}{\mathbb{R}}
\DeclareMathOperator{\1}{\mathbf{1}}

\let\oldproof\proof
\renewcommand{\proof}{\color{blue}\oldproof}

%\NewEnviron{killcontents}{}\let\proof\killcontents\let\endproof\endkillcontents



\begin{document}
\begin{center}
{\Large\textbf STAT 381-383-385 (2021)\hspace{0.1cm} Theorem Page}\medbreak
\large{Alex Sheng}
\end{center}

\vspace{0.2 cm}
\tableofcontents

\newpage
\section{Measure Theory}

\subsection{Measures}

\begin{prop}\label{Prop 1.1}
The following two properties for a finitely additive measure $\mu$ are equivalent:\begin{itemize}
    \item $\mu$ is downward continuous: if $\{E_i\}_{i\geq1}$ is a decreasing sequence of sets such that $\mu(E_k)<\infty$ for some $k$, then $\mu(E_i)\to\mu(\cap_{i\geq1}E_i).$
    \item $\mu$ is $\sigma$-additive.
\end{itemize}
\end{prop}

\begin{prop}\label{Prop 1.2}
Let $\mu$ be an outer measure on $\Omega$. Let $\mathcal{F}$ be the collection of all $\mu$-measurable subsets of $\Omega$. Then $\mathcal{F}$ is a $\sigma$-algebra and $\mu$ is a measure on $\mathcal{F}$.
\end{prop}

\begin{remark}
Hands on construction of Jordan measure and Lebesgue measure.\medbreak
A subset of $\mathbb{R}^d$ is elementary if it is a finite union of finite boxes. Jordan measure $\mu$ of elementary set $E$ is simply the volume of $E$. Jordan measure of arbitrary $A\subset\mathbb{R}^d$ is defined by
\[\mu(A)=\inf\{\mu(E):A\subset E\textrm{ and }E\textrm{ is elementary}\}.\]
A set $A\subset\mathbb{R}^d$ is called Jordan measurable if for any $\epsilon>0$ there exists elementary sets $E$ and $F$ such that $E\subset A\subset F$ and $\mu(F\setminus E)<\epsilon$. Then the collection of all Jordan measurable sets is a Boolean algebra and $\mu$ is a finitely additive measure defined on it.\medbreak
For any $A\subset\mathbb{R}^d$ define
\[\mu^\ast(A)=\inf\left\{\sum_{n\geq1}\mu(B_n):E\subset\bigcup_{i\geq1}B_i\textrm{ and }B_n\textrm{ are boxes in }\mathbb{R}^d\right\}.\]
Then $\mu^\ast$ is an outer measure. A subset $A\subset\mathbb{R}^d$ is called Lebesgue measurable if for any $\epsilon>0$ there exists a countable union of boxes $C=\cup_{i\geq1}B_i$ such that $E\subset C$ and $\mu^\ast(C\setminus E)<\epsilon$. Without invoking \hyperref[Prop 1.1]{Prop 1.2}, we will show that\begin{itemize}
    \item $\mu^*$ extends $\mu$: $\mu^\ast(E)=\mu(E)$ for all Jordan measurable sets $E$,
    \item the collection of all Lebesgue measurable sets $\mathcal{L}$ is a $\sigma$-algebra, and
    \item $\mu^\ast|_\mathcal{L}$ is a measure.
\end{itemize}
Such extension $\mu$ is the Lebesgue measure on $\mathbb{R}^d$. $\mu^\ast$ is the Lebesgue outer measure.\medbreak
We first show that $\mu$ is downward continuous. Indeed, let $\{E_k\}_{k\geq1}$ be a decreasing sequence of elementary sets and $\cap_{k\geq1}E_k=\varnothing$. Shrink a bit each side of $E_k$ and find a closed elementary $F_k\subset E_k$ such that $\mu(E_k\setminus E_k)\leq2^{-k-1}\epsilon$ for some $\epsilon>0$. Then
\[\mu(E_k)-\mu(\bigcap_{i=1}^kF_i)\leq\mu(E_k\setminus\bigcap_{i=1}^kF_i)=\mu(\bigcap_{i=1}^k(E_k\setminus F_i))\leq\sum_{i=1}^k\mu(E_i\setminus F_i)\leq\frac{\epsilon}{2}.\]
Since $\cap_{i\geq1}F_i=\varnothing$, we have $\lim_{k\to\infty}\mu(E_k)=0$ as desired.\medbreak
Now we will prove that $\mu^\ast(A)=\mu(A)$ for any Jordan measurable set $A$. Since $\mu^*$ takes infimum over countable cover of boxes and $\mu$ takes infimum over finite union of boxes, $\mu^\ast(A)\leq\mu(A)$. For the reverse inequality, first let $A$ be elementary, then there exists boxes $\{B_i\}_{i\geq1}$ such that $A\subset\cup_{i\geq1}B_i$ and $\sum_{i\geq1}\mu(B_i)\leq\mu^\ast(A)+\epsilon$ for any $\epsilon>0$. Let $E_n=A\setminus\cup_{i=1}^nB_i$ which is elementary and $\cap_{i\geq1}E_i=\varnothing$. By downward continuity of $\mu$,
\[\mu(A)\leq\mu(E_n)+\mu(\bigcup_{i=1}^nB_i)\leq\mu(E_n)+\sum_{i=1}^n\mu(B_i)\leq\mu(E_n)+\mu^\ast(A)+\epsilon.\]
Taking $n\to\infty$ the result follows. Now for arbitrary Jordan measurable $A$ we have elementary $E$ such that $E\subset A$ and $\mu(A)\leq\mu(E)+\epsilon=\mu^\ast(E)+\epsilon\leq\mu^\ast(A)+\epsilon$. Hence $\mu^\ast(A)=\mu(A)$.\medbreak
Trivially $\varnothing\in\mathcal{L}$. We only need to check for complement-stability and countable $\cup$-stability.\begin{itemize}
    \item Let $A=\cup_{i\geq1}A_i$ with $A_i\in\mathcal{L}$ for all $i$. For any $\epsilon>0$ let $C_i$ be a countable union of boxes with $A_i\subset C_i$ and $\mu^\ast(C_i\setminus A_i)\leq2^{-i}\epsilon$ for all $i$. Then $\cup_{i\geq1}C_i$ is a countable union of boxes with $A\subset\cup_{i\geq1}C_i$ and
    \[\mu^\ast(\bigcup_{i\geq1}C_i\setminus A)\leq\sum_{i\geq1}\mu^\ast(C_i\setminus A)\leq\sum_{i\geq1}\mu^\ast(C_i\setminus A_i)\leq\epsilon,\]
    which implies that $A\in\mathcal{L}$.
    \item Let $B$ be any box and $E\in\mathcal{L}$. There exists a countable union of boxes $C_n$ such that $E\subset C_n$ and $\mu^\ast(C_n\setminus E)<n^{-1}$. By countable $\cup$-stability $D=\cup_{n\geq1}B\setminus C_n\in\mathcal{L}$. Since $(B\cap E^c)\cap D^c$
    \[\mu^\ast((B\cap E^c)\cap D^c)\leq\mu^\ast((B\cap E^c)\setminus(B\cap C_n^c))=\mu^\ast(C_n\setminus E)<n^{-1},\]
    it is clear that $(B\cap E^c)\cap D^c\in\mathcal{L}$. Hence $B\setminus E=D\cup((B\cap E^c)\cap D^c)\in\mathcal{L}$. Since $E^c=\cup_{n\geq1}[-n,n]^d\setminus E$, by countable $\cup$-stability $E^c\in\mathcal{L}$.
\end{itemize}
To check $\mu^\ast$ is a measure on $\mathcal{L}$, we only need to show its $\sigma$-additivity for $\{E_i\in\mathcal{L}\}_{i\geq1}$. It suffice to assume that $E_i$ is bounded for all $i$, since we can partition $\mathbb{R}^d$ into countable disjoint unions of bounded subsets. Since
\[\sum_{i=1}^n\mu^\ast(E_i)=\mu^\ast(\bigcup_{i=1}^nE_i)\leq\mu^\ast(\bigcup_{i\geq1}E_i)\leq\sum_{i\geq1}\mu^\ast(E_i),\]
when $n\to\infty$ the $\sigma$-additivity follows from finite additivity. Furthermore, it suffice to prove the finite additivity for countable intersections of elementary sets, since they approximate Lebesgue measurable sets. Indeed, for $E\in\mathcal{L}$ we are able to find countable union of elementary sets $C_n$ such that $E\subset C_n$ and $\mu^\ast(C_n\setminus E)<n^{-1}$, hence $\mu^\ast(\cap_{n\geq1}C_n\setminus E)=0$. Since countable intersections can be approximated by finite intersections, by finite additivity the result follows.
\end{remark}

\begin{prop}\label{Dynkin’sπ-λTheorem}\textup{\textbf{(Dynkin's $\pi$-$\lambda$ Theorem)}} Let $\Pi$ be a $\pi$-system and $\Lambda$ a $\lambda$-system. If $\Pi\subset\Lambda$ then $\sigma(\Pi)\subset\Lambda$.
\end{prop}

\begin{prop}\textup{\textbf{(Carathéodory's Extension Theorem)}} Every pre-measure on a Boolean algebra $\mathcal{B}$ can be extended to a measure on the $\sigma$-algebra $\sigma(\mathcal{B})$ generated by $\mathcal{B}$. The extension is unique if the pre-measure is $\sigma$-finite.
\end{prop}

\begin{prop}
properties of measurable functions
\end{prop}

\subsection{Integration}

\begin{prop}
Non-negative measurable functions can be pointwise approximated by a non-decreasing sequence of simple functions.
\end{prop}

\begin{prop}
properties of integration
\end{prop}

\begin{prop}
Let $\{f_n\}_{n\geq1}$ be a sequence of measurable functions on measure space $(\Omega,\mathcal{F},\mu)$
\begin{itemize}
    \item \textup{\textbf{(Monotone Convergence Theorem})} If $f_n\geq0$ and $f_n\leq f_{n+1}$ for all $n\geq1$, then \[\int_\Omega\lim_{n\to\infty}f_n\,d\mu=\lim_{n\to\infty}\int_\Omega f_n\,d\mu.\]
    \item \textup{\textbf{(Fatou's lemma)}} If $f_n\geq0$ for all $n\geq1$, then
    \[\int_\Omega\liminf_{n\to\infty}{f_n}\,d\mu\leq\liminf_{n\to\infty}\int_\Omega f_n\,d\mu.\]
    \item \textup{\textbf{(Dominated Convergence Theorem)}}\label{Dominated Convergence Theorem} If $\abs{f_n(x)}\leq g(x)$ for all $n\geq1$ and some integrable function $g$, then
    \[\int_\Omega\lim_{n\to\infty}f_n\,d\mu=\lim_{n\to\infty}\int_\Omega f_n\,d\mu.\]
\end{itemize}
\end{prop}

\begin{remark} (Feynman's technique) \textit{Let $(\Omega,\mathcal{F},\mu)$ be a measure space and $U\subset\mathbb{R}$ an open set. For $f:U\times\Omega\to\mathbb{R}$ suppose\begin{itemize}
    \item $x\mapsto f(t,\omega)$ is integrable for every $t\in U$,
    \item $t\mapsto f(t,\omega)$ is differentiable with respect to $t$ for every $\omega\in\Omega$, and
    \item there is an integrable $h:\Omega\to\mathbb{R}$ such that $\abs{\frac{d}{dt}f(t,\omega)}\leq h(\omega)$ for all $(t,\omega)\in U\times \Omega$.
\end{itemize}
Then
\[\frac{d}{dt}\int_\Omega f(t,\omega)\,d\mu(\omega)=\int_\Omega\frac{d}{dt}f(t,\omega)\,d\mu(\omega).\]}Indeed, let $s_n$ be a sequence of real numbers that tends to 0 and $g_n(t,\omega)=\frac{1}{s_n}(f(t+s_n,\omega)-f(t,\omega))$. Then $g_n(t,\omega)\to\frac{d}{dt}f(t,\omega)$ and by Mean Value Theorem, $g_n(t,\omega)\leq\sup_{t\in[t,t+s_n]}\abs{\frac{d}{dt}f(t,\omega)}\leq h(\omega)$. The result follows from \hyperref[Dominated Convergence Theorem]{Dominated Convergence Theorem}:
\begin{align*}
\int_\Omega\frac{d}{dt}f(t,\omega)\,d\mu(\omega)&=\int_\Omega\lim_{n\to\infty}g_n\,d\mu(\omega)\\
&=\lim_{n\to\infty}\int_\Omega g_n\,d\mu(\omega)=\frac{d}{dt}\int_\Omega f(t,\omega)\,d\mu(\omega).
\end{align*}
\end{remark}

\subsection{Measure and integration on product spaces}

\begin{prop}\label{Prop 1.9}
Let $(X,\mathcal{X},\mu)$ and $(Y,\mathcal{Y},\nu)$ be measure spaces, and $f:X\times Y\to[0,\infty]$ a measurable function. For any $x\in X$:\begin{itemize}
    \item Slice $E_x=\{y\in Y:(x,y)\in E\}\in\mathcal{Y}$ for any $E\in\mathcal{X}\otimes\mathcal{Y}$,
    \item Function $g:Y\to[0,\infty]$ by $y\mapsto f(x,y)$ is measurable, and
    \item Function $h:Y\to[0,\infty]$ by $y\mapsto\int_Yf(x,y)\,d\nu(y)$ is measurable, if $\nu$ is $\sigma$-finite.
\end{itemize}
\end{prop}

\begin{prop}\label{Prop 1.10}
Let $(X,\mathcal{X},\mu)$ and $(Y,\mathcal{Y},\nu)$ be measure spaces, then there exists the product measure $\sigma=\mu\otimes\nu$ on $\mathcal{X}\otimes\mathcal{Y}$. Moreover $\sigma$ is unique if the measure spaces are $\sigma$-finite.
\end{prop}

\begin{prop}\textup{\textbf{(Fubini-Tonelli Theorem)}}
Let $(X,\mathcal{X},\mu)$ and $(Y,\mathcal{Y},\nu)$ be $\sigma$-finite measure spaces. If $f:X\times Y\to[0,\infty]$ is measurable, then
\[\int_{X\times Y}f\,d(\mu\otimes\nu)=\int_X\,d\mu(x)\int_Yf(x,y)\,d\nu(y)=\int_Y\,d\nu(y)\int_Xf(x,y)\,d\mu(x).\]
\end{prop}

\newpage
\section{Basic Proabbility}

\subsection{Basic notions}

\begin{prop}\label{Prop 2.1}
If $F$ is the distribution function of random variable $X$, then $F$ is:\begin{itemize}
    \item non-decreasing,
    \item right-continuous: $\lim_{\Delta t\to0}F(\Delta t+t)=F(t)$,
    \item $\lim_{t\to-\infty}F(t)=0$, and $\lim_{t\to\infty}F(t)=1$.
\end{itemize}
Moreover, $F$ determines the law $\mu_X$ of $X$ uniquely. Conversely if $F:\mathbb{R}\to[0,1]$ satisfies the listed properties, then there is a probability space $(\Omega, \mathcal{F},\mathbb{P})$ and a random variable $X$ such that $F$ is the distribution function of $X$.
\end{prop}

\begin{prop}\textup{\textbf{(Jensen's Inequality) }}Let $X$ be a random variable on measure space $(\Omega,\mathcal{F},\mathbb{P})$. If $X\in L^1(\mathbb{P})$ and $\phi:I\to\mathbb{R}$ is convex on some $I\subset\mathbb{R}$: for all $x,y\in I$ and $t\in[0,1]$,
\[\phi(tx+(1-t)y)\leq t\phi(x)+(1-t)\phi(y).\]
Then
\[\phi(\E[X])\leq\E[\phi(X)].\]
\end{prop}

\begin{prop}
Let $X$ and $Y$ be random variables on measure space $(\Omega,\mathcal{F},\mathbb{P})$ and $p,q\in[1,\infty]$ be such that $pq=p+q$.
\begin{itemize}
    \item \textup{\textbf{(Hölder's Inequality)}} If $X\in L^p(\mathbb{P})$ and $Y\in L^q(\mathbb{P})$, then $\norm{XY}_1\leq\norm{X}_p\norm{Y}_q.$
    \item \textup{\textbf{(Minkowski's Inequality)}} If $X\in L^p(\mathbb{P})$ and $Y\in L^p(\mathbb{P})$, then $\norm{X+Y}_{p}\leq\norm{X}_p\norm{Y}_p.$
\end{itemize}
\end{prop}

\begin{prop}
Let $X$ be a random variable on measure space $(\Omega,\mathcal{F},\mathbb{P})$.
\begin{itemize}
    \item \textup{\textbf{(Markov's Inequality)}} If $X$ is non-negative and $t\geq0$, then $t\mathbb{P}(X\geq t)\leq\mathbb{E}(X).$
    \item \textup{\textbf{(Chebyshev's Inequality)}} For any $t\geq0$, $t^2\mathbb{P}(\abs{X-\mathbb{E}(X)}\geq t)\leq\Var(X).$
    \item \textup{\textbf{(Chernoff bound)}} Given $m_X(\lambda)=\mathbb{E}(e^{\lambda X})$ the moment-generating function of $X$ and $\lambda>0$,
    \[e^{\lambda t}\mathbb{P}(X\geq t)\leq m_X(\lambda).\]
\end{itemize}
\end{prop}

\subsection{Independence and tail events}

\begin{prop}The followings are sufficient conditions for independence.\begin{itemize}
    \item If $\pi$-systems $\Pi_1$ and $\Pi_2$ are  independent, then $\mathcal{A}_1=\sigma(\Pi_1)$ and $\mathcal{A}_2=\sigma(\Pi_2)$ are independent.
    \item If $\mathbb{P}(X_1\leq t_1\textrm{ and }X_2\leq t_2)=\mathbb{P}(X_1\leq t_1)\mathbb{P}(X_2\leq t_2)$ for all $x_1,x_2\in\overline{\mathbb{R}}$, then $X_1,X_2$ are independent random variables.
    \item Let $f_1$ and $f_2$ be measurable functions $\mathbb{R}\to\mathbb{R}$ and $X_1,X_2$ independent random variables. Then $f_2(X_1)$ and $f_2(X_2)$ are independent.
\end{itemize}
\end{prop}

\begin{prop}
The followings are consequences of independence.\begin{itemize}
    \item If $X_1$ and $X_2$ be independent random variables with laws $\mu_1$ and $\mu_2$, then $(X_1,X_2)$ has law $\mu_1\times\mu_2$.
    \item For independent random variables $X_1$ and $X_2$, if $X_1\geq0$ or $\E(\abs{X_i})\leq\infty$ for $i=1,2$, then
    \[\E(X_1X_2)=\E(X_1)\E(X_2).\]
\end{itemize}
\end{prop}

\begin{prop}\textup{\textbf{(Hoeffding bound)}} Let $X_1,X_2,\cdots,X_n$ be a sequence of independent random variables such that $X_i\in[a_i,b_i]$ almost surely for all $i\in[1,n]$. Then for any $t\geq0$,
\[\mathbb{P}\left(\sum_{i=1}^n(X_i-\E(X_i))\geq t\right)\leq\exp{\frac{-Ct^2}{\sum_{i=1}^n(b_i-a_i)^2}}\]
for some constant $C$.
\end{prop}

\begin{prop}Let $(\Omega,\mathcal{F},\mathbb{P})$ be the measure space and $\{A_n\}_{n\geq1}$ a sequence of events.\begin{itemize}
    \item\textup{\textbf{(Borel-Cantelli Lemma I)}} If $\sum_{n\geq1}\mathbb{P}(A_n)<\infty$, then
\[\mathbb{P}(\limsup_{n\to\infty}A_n)=\mathbb{P}(\omega:\omega\in A\textrm{ infinitely often})=0.\]
    \item\textup{\textbf{(Borel-Cantelli Lemma II)}} If $\{A_n\}$ is pairwise independent and $\sum_{n\geq1}\mathbb{P}(A_n)=\infty$, then
    \[\mathbb{P}(\limsup_{n\to\infty}A_n)=1.\]
\end{itemize}
\end{prop}

\begin{remark}\textup{\textbf{Infinite Monkey Theorem:}} Let
\end{remark}

\begin{prop}\textup{\textbf{(Kolmogorov's 0-1 Law)}} Let $\{X_i\}_{i\geq1}$ be a sequence of independent random variables and $\mathcal{T}$ the tail $\sigma$-algebra of $\{X_i\}$. Then $\mathcal{T}$ is trivial: $\mathbb{P}(A)=0$ or $1$ for all $A\in\mathcal{T}$.
\end{prop}

\begin{remark}
Consider the collection of events $\mathcal{E}$ that the occurrence is invariant under finite permutation. $\mathcal{E}$ is a $\sigma$-algebra called exchangeable $\sigma$-algebra. A stronger version of the above theorem is \textbf{Hewitt-Savage 0-1 Law}:\medbreak
\textit{Let $\{X_i\}_{i\geq1}$ be a sequence of independent and identically distributed random variables and $\mathcal{E}$ the exchangeable $\sigma$-algebra. Then $\mathcal{E}$ is trivial: $\mathbb{P}(A)=0$ or $1$ for all $A\in\mathcal{T}$.}
\end{remark}

\subsection{Convergence of random variables}

\begin{prop}
The following statements regarding the convergence of random variables holds:\begin{itemize}
    \item If $X_n\xrightarrow{a.s.}X$ or $X_n\xrightarrow{L^p}X$ for some $p$, then $X_n\xrightarrow{\mathbb{P}}X$. If $X_n\xrightarrow{\mathbb{P}}X$ then $X_n\xrightarrow{d}X$.
    \item If $X_n\xrightarrow{L^p}X$ for some $p$, $X_n\xrightarrow{\mathbb{P}}X$, or the sum $\sum_{n\geq1}\mathbb{P}(\abs{X_n-X}>\epsilon)<\infty$ for any $\epsilon>0$, then there is a subsequence $X_{n_k}\xrightarrow{a.s.}X$.
    \item If $X_n\xrightarrow{d}C$ for some constant $C$ then $X_n\xrightarrow{\mathbb{P}}X$.
    \item If $\{X_n\}$ is uniformly integrable, then $X_n\xrightarrow{a.s.}X$ implies $X_n\xrightarrow{L^1}X$; if $\{X_n\}$ is uniformly bounded, then $X_n\xrightarrow{\mathbb{P}}X$ implies $X_n\xrightarrow{L^p}X$ for all $p$.
    \item If $1\leq q\leq p$, then $X_n\xrightarrow{L^p}X$ implies $X_n\xrightarrow{L^q}X$.
    \item $X_n\xrightarrow{d}X$ if and only if $X_n\xrightarrow{weak}X$.
\end{itemize}
Or to put in a diagram:
\begin{center}\begin{tikzcd}
\begin{matrix}
    L^p\Rightarrow L^q\\1\leq q\leq p 
\end{matrix} \arrow[rd, Rightarrow] \arrow[dd, "subsequence" description, bend left]       &                                                                                                                      &                                                            &    \\
                                                                                  & \mathbb{P} \arrow[r, Rightarrow, bend right] \arrow[lu, "\textrm{uniform bounded}"', bend right] \arrow[ld, "\textrm{subsequence or sum}", bend left] & d \arrow[r, Leftrightarrow] \arrow[l, "X_n\xrightarrow{d} C"', bend right] & weak\\
a.s. \arrow[ru, Rightarrow] \arrow[uu, "\textrm{uniform integrable (to }L^1\textrm{)}", bend left=49] &                                                                                                                      &                                                            & 
\end{tikzcd}\end{center}
\end{prop}

\begin{prop}\textup{\textbf{(Portemanteau lemmas)}} Let $(\mathcal{S},\rho)$ be a polish space and $\{\mu_n\}$ a sequence of probability measures on $\mathcal{S}$, then the following statements are equivalent.\begin{itemize}
    \item $\mu_n\xrightarrow{weak}\mu$.
    \item For every bounded $f:\mathcal{S}\to\mathbb{R}$ that is Lipschitz continuous: for any $x,y\in\mathcal{S}$ there exists $L<\infty$ such that $\abs{f(x)-f(y)}\leq L\rho(x,y)$,
    \[\int_\mathcal{S}f\,d\mu_n\to\int_\mathcal{S}f\,d\mu.\]
    \item $\limsup_{n\to\infty}\mu_n(F)\leq\mu(F)$ and $\liminf_{n\to\infty}\mu_n(E)\geq\mu(A)$ for every closed $F$ and open $E$.
    \item $\mu_n(A)\to\mu(A)$ for all measurable $A\subset\mathcal{S}$ with $\mu(\partial A)=0$.
    \item For every bounded measurable $f:\mathcal{S}\to\mathbb{R}$ that is continuous almost everywhere with respect to $\mu$,
    \[\int_\mathcal{S}f\,d\mu_n\to\int_\mathcal{S}f\,d\mu.\]
\end{itemize}
\end{prop}

\begin{remark}
It follows that if $\int_\mathcal{S}f\,d\mu=\int_\mathcal{S}f\,d\nu$ for probability measures $\mu$ and $\mu$  and all bounded continuous $f:\mathcal{S}\to\mathbb{R}$, then $\mu=\nu$. Indeed, by Portemanteau lemmas $\mu$ and $\nu$ thus agree on all open sets, thus by Dynkin's $\pi$-$\lambda$ Theorem the uniqueness follows.
\end{remark}

\begin{prop}\textup{\textbf{(Slutsky's Theorem) }}If $X_n\xrightarrow{d}X$ and $Y_n\xrightarrow{\mathbb{P}}c$ for some constant $c$, then $X_n+Y_n\xrightarrow{d}X+c$ and $X_nY_n\xrightarrow{d}Xc$.
\end{prop}

\begin{remark}
Metrization of space of probability measures: let $(\mathcal{S},\rho)$ be a Polish space and define $A^\epsilon=\{x\in\mathcal{S}:\rho(x,y)<\epsilon\textrm{ and }y\in A\}$ for any $A\subset S$. For any two probability measures $\mu,\nu$ on $\mathcal{S}$, the \textbf{Lévy-Prokhorov metric}
\[d(\mu,\nu)=\inf\{\epsilon>0:\mu(A)\leq\nu(A^\epsilon)+\epsilon\textrm{ and }\nu(A)\leq\mu(A^\epsilon)+\epsilon\textrm{ for all }A\in\mathfrak{B}(\mathcal{S})\}\]
metrizes weak convergence of probability measures: $\mu_n\xrightarrow{weak}\mu$ if and only if $d(\mu_n,\mu)\to0$.
\end{remark}

\subsection{Laws of large numbers}

\begin{prop}\textup{\textbf{($L^2$ Weak Law) }}Let $\{X_i\}_{i\geq1}$ be a sequence of uncorrelated and identically distributed random variables with finite variance for all $i$. Then
\[\frac{1}{n}\sum_{i=1}^nX_n\xrightarrow{L^2}\E(X_1)\]
\end{prop}

\begin{remark}\textup{\textbf{Coupon Collector's Problem:}} Let
\end{remark}

\begin{prop}\textup{\textbf{(Weak Law of Triangular Array) }}Let $a_n$ be a sequence of positive integers such that $a_n\to\infty$. Let $\{X_{n,k}\}_{1\leq k\leq n}$ be a sequence of independent random variables and truncation $\overline{X}_{n,k}=X_{n,k}\1_{\{\abs{X_{n,k}\leq a_n}\}}$. If as $n\to\infty$\begin{itemize}
    \item $\sum_{k=1}^n\mathbb{P}(\abs{X_{n,k}}>a_n)\to0$, and
    \item $b_n^{-2}\sum_{k=1}^n\E(\overline{X}_{n,k}^2)\to0$,
\end{itemize}
then
\[b_n^{-1}\sum_{k=1}^n(X_{n,k}-\E(\overline{X}_{n,k}))\xrightarrow{\mathbb{P}}0\]
\end{prop}

\begin{prop}\textup{\textbf{(Weak Law of Large Numbers) }}Let $\{X_i\}_{i\geq1}$ be a sequence of independent and identically distributed random variables such that $n\mathbb{P}(\abs{X_1}>n)\to0$ as $n\to\infty$. Then
\[\frac{1}{n}\sum_{i=1}^n X_i\xrightarrow{\mathbb{P}}\E(X_1)\]
\end{prop}

\begin{prop}\textup{\textbf{(Strong Law of Large Numbers) }}Let $\{X_i\}_{i\geq1}$ be a sequence of pairwise independent and identically distributed random variables such that $\mathbb{E}(\abs{X_1})<\infty$. Then
\[\frac{1}{n}\sum_{i=1}^nX_i\xrightarrow{a.s.}\mathbb{E}(X_1).\]
\end{prop}

\subsection{Tightness and characteristic functions}

\begin{prop}
Let $\{X_n\}_{n\geq1}$ be a sequence of random variables.
\begin{itemize}
\item If $\{X_n\}$ be a tight family and $\{c_n\}_{n\geq1}$ a sequence of constants converging to 0. Then,\begin{itemize}
    \item \textup{\textbf{(Helly's Selection Theorem)}} there is a subsequence $X_{n_k}\xrightarrow{d}X$, and
    \item $c_nX_n\xrightarrow{\mathbb{P}}0$.
\end{itemize}
\item Conversely if $X_n\xrightarrow{d}X$, then $\{X_n\}$ is a tight family.
\end{itemize}
\end{prop}

\begin{prop}
Let $\{\mu_n\}_{n\geq1}$ be a sequence of probability measures on a Polish space $(\mathcal{S},\rho)$.\begin{itemize}
    \item If $\mu_n\xrightarrow{weak}\mu$, then $\{\mu_n\}$ is tight.
    \item \textup{\textbf{(Prokhorov's Theorem)}} If $\{\mu_n\}$ is tight, then there is a subsequence $\mu_{k_n}\xrightarrow{weak}\mu$.
\end{itemize}
\end{prop}

\begin{prop}
The following properties of characteristic functions hold.
\begin{itemize}
    \item $\abs{\phi(t)}\leq\E(\abs{e^{itX}})=1$.
    \item $\phi(t)$ is uniformly continuous: $\abs{\phi(t+h)-\phi(t)}\leq\E(\abs{e^{ihX}-1})$.
    \item if $X$ and $Y$ are independent, then $\phi_{X+Y}=\phi_X\phi_Y$.
\end{itemize}
\end{prop}

\begin{prop}
Let $\phi$ be a function and $t_1<t_2<\cdots<t_n$. Let $A_{ij}=\phi(t_i-t_j)$ be a $n\times n$ matrix.\begin{itemize}
    \item If $\phi$ is a characteristic function of some random variable, then the following properties hold:\begin{itemize}
        \item A is Hermitian: $A^*=\overline{A^T}=A$.
        \item A is positive semi-definite: for any $\forall v\in\mathbb{C}^n:v^TAv\geq0$.
        \item $\phi$ is continuous at 0 with $\phi(0)=1$.
    \end{itemize}
    \item \textup{\textbf{(Bochner's Theorem)}} Conversely if $\phi$ satisfies the listed properties, then it is the characteristic function of some random variable.
\end{itemize}
\end{prop}

\begin{remark}
example: cf of N(mu,sigma)
\end{remark}

\begin{prop}\textup{\textbf{(Inversion Formula)}} For any random variable $X$ and $\theta>0$ define
\[f_\theta(X)=\frac{1}{2\pi}\int_{-\infty}^\infty e^{itX-\theta t^2}\phi_X(t)\,dt.\]
Then for any bounded continuous function $g:\R\to\R$,
\[\E[g(X)]=\lim_{\theta\to0}\int_{-\infty}^\infty g(x)f_\theta(x)\,dx.\]
\end{prop}

\begin{remark}
Two random variables have the same law if and only if they have the same characteristic function. Indeed, if $X$ and $Y$ have the same law, then they have the same $k$-th moment $\E[X^k]=\E[Y^k]$ for all $k$. The characteristic function $\E[e^{iXt}]$ can be decomposed to sum of $k$th moments:
\begin{align*}
\E[e^{iXt}]&=\E(1+iXt-\frac{X^2t^2}{2!}+\cdots+\frac{i^nX^nt^n}{n!}+\cdots)\\&=1+it\E[X]-\frac{t^2\E[X^2]}{2!}+\cdots+\frac{t^n\E[X^n]}{n!}+\cdots.
\end{align*}
On the other hand, if $X$ and $Y$ have the same characteristic function, then by Inversion Formula they have the same law.
\end{remark}

\begin{remark}
Suppose $\int_{-\infty}^\infty\abs{\phi_X(t)}\,dt<\infty$, then by Inversion Formula the probability density function of $X$ is given by
\[f(x)=\frac{1}{2\pi}\int_{-\infty}^\infty e^{-itx}\phi_X(t)\,dt.\]
\end{remark}

\begin{prop}\textup{\textbf{(Lévy's Continuity Theorem)}} A sequence of random variables $\{X_n\}$ converges to $X$ in distribution if and only if their characteristic functions $\{\phi_{X_n}\}$ converges to $\phi_X$ pointwise.
\end{prop}

\begin{remark}
\textup{\textbf{Cramér-Wold Theorem:}} Let $\{X_n\}_{n\geq1}$ be a sequence of random vectors in $\mathbb{R}^n$, then $X_n\xrightarrow{d}X$ if and only if $tX_n\xrightarrow{d}tX$ for any $t\in\mathbb{R}^n$. Indeed, one direction is a consequence of the fact that all finite-dimensional linear maps are continuous; the other direction follows from Lévy's Continuity Theorem.
\end{remark}

\begin{prop}\textup{\textbf{(Central Limit Theorem)}} Let $\{X_i\}_{i\geq1}$ be a sequence of independent and identically ditributed random variables with $\E(X_1)=\mu$ and $\Var(E_1)=\sigma^2<\infty$. Then
\[\frac{\sqrt{n}}{\sigma}\left(\frac{\sum_{i=1}^nX_i}{n}-\mu\right)\xrightarrow{d}\mathcal{N}(0,1).\]
\end{prop}

\begin{remark}
The above Central Limit Theorem is the classical result. A stronger one that reduces the identical distribution condition is \textbf{Lindeberg-Feller Central Limit Theorem}:\medbreak
\textit{Let $\{k_n\}_{n\geq1}$ be a sequence of positive integers with $k_n\to\infty$. Let $\{X_{n,i}\}_{1\leq i\leq k_n}$ be a collection of independent random variables and denote $\mu_{n,i}=\E(X_{n,i})$, $\sigma_{n,i}^2=\Var(X_{n,i})$, and $S_n^2=\sum_{i=1}^{k_n}\sigma_{n,i}^2$. If $\{X_{n,i}\}$ satisfies Lindeberg condition:}
\[\forall\epsilon>0:\frac{1}{S_n^2}\sum_{i=1}^{k_n}\E\left[(X_{n,i}-\mu_{n,i})^2\textbf{1}_{\{\abs{X_{n,i}-\mu_{n,i}}\geq\epsilon S_n\}}\right]\to0.\]
\textit{Then}
\[\frac{1}{S_n}\sum_{i=1}^{k_n}(X_{n,i}-\mu_{n,i})\xrightarrow{d}\mathcal{N}(0,1).\]
As a corollary to the Lindeberg-Feller is \textbf{Lyapunov Central Limit Theorem}:\medbreak
\textit{Let $\{X_i\}_{i\geq1}$ be a sequence of independent random variables with $\E(X_i)=\mu_i$, $\sigma_i^2=\Var(X_i)<\infty$ and $S_n^2=\sum_{i=1}^n\sigma_i^2$. If $\{X_i\}$ satisfies Lyapunov condition:}
\[\exists\delta>0:\frac{1}{S_n^{2+\delta}}\sum_{i=1}^n\E\left[\abs{X_i-\mu_i}^{2+\delta}\right]\to0.\]
\textit{Then}
\[\frac{1}{S_n}\sum_{i=1}^n(X_i-\mu_i)\xrightarrow{d}\mathcal{N}(0,1).\]
\end{remark}

\newpage
\section{Construction of Brownian Motion}

\begin{prop}
The following statements regarding finite-dimensional projection maps and distributions hold:\begin{itemize}
    \item A probability measure $\mu$ on $C[0,1]$ is determine by its finite-dimensional distributions.
    \item The finite-dimensional projection maps generate $\mathfrak{B}(C[0,1])$.
    \item If $\{\mu_n\}_{n\geq1}$ is a tight family of probability measures on $C[0,1]$ whose finite-dimensional distributions $F_n\to F$, then the $\mu_n\xrightarrow{weak}\mu$ for some $\mu$ uniquely determined by $F$.
\end{itemize}
\end{prop}

\begin{prop}
A sequence $\{X_n\}_{n\geq1}$ of $C[0,1]$-valued random variables is tight if and only the following two conditions hold:\begin{itemize}
    \item For any $\epsilon>0$ there exists some $a>0$ such that $\mathbb{P}(\abs{X_n(0)}>a)<\epsilon$ for all $n$.
    \item For any $\epsilon,\eta>0$ there exists some $\delta>0$ and $N(\epsilon,\eta)>0$ such that $\mathbb{P}(w_{X_n}(\delta)<\eta)\leq\epsilon$ when $n>N$ where $w_{X_n}(\delta)=\sup\{\abs{X_n(s)-X_n(t)}:\abs{s-t}<\delta\}$.
\end{itemize}
\end{prop}

\begin{prop}\textup{\textbf{(Donsker's Invariant Principle)}} Brownian motion exists as the weak limit $C[0,1]$-valued random variable such that $B(0)=0$ and for any $m$ and $0<t_1<t_2<\cdots<t_m\leq1$ the random vector $(B(t_1),B(t_2),\cdots,B(t_m))\sim\mathcal{N}_m(\mathbf{0},\Sigma)$ where
\[\Sigma_{ij}=\Cov(B(t_i),B(t_j))=\min\{t_i,t_j\}.\]
\end{prop}



\end{document}