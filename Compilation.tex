\documentclass[hidelinks,11pt]{article}

\usepackage[margin=1in]{geometry}
\usepackage{amsmath,amsthm,amsfonts}
\usepackage[utf8]{inputenc}
\usepackage{amssymb}
\usepackage[mathscr]{eucal}
\usepackage{graphicx}
\usepackage{listings}
\usepackage{xcolor}
\usepackage[OT1]{fontenc}
\usepackage{physics}
\usepackage{tikz-cd}
\usepackage{xpatch}
\usepackage{nicefrac}
\usepackage{mathtools}
\usepackage{environ}

\setlength\parindent{0pt}

\newtheoremstyle{dotless}{}{}{\itshape}{}{\bfseries}{}{ }{}

\theoremstyle{definition}
\newtheorem{defin}{DEF}

\theoremstyle{dotless}
\newtheorem{prop}{PROP}[section]
\newtheorem*{corollary}{Corollary}

\newtheoremstyle{named}{}{}{\upshape}{}{\bfseries}{}{.5em}{\thmname{#1} \thmnote{#3}}
\theoremstyle{named}
\newtheorem*{prof}{Proof}

\theoremstyle{remark}
\newtheorem*{remark}{Remark}

\usepackage{hyperref}
\hypersetup{colorlinks=true,linkcolor=magenta}

\DeclareMathOperator{\Var}{Var}
\DeclareMathOperator{\E}{\mathbb{E}}
\DeclareMathOperator{\R}{\mathbb{R}}
\DeclareMathOperator{\1}{\mathbf{1}}



\begin{document}
\begin{center}
{\Large\textbf STAT 381-383-385\hspace{0.2cm} (2021)}\medbreak
\large{Alex Sheng}
\end{center}

\vspace{0.2 cm}
\tableofcontents

\newpage
\section{Measure Theory}

\subsection{Measures}

\begin{prop}\label{Prop 1.1}
The following two properties for a finitely additive measure $\mu$ are equivalent:\begin{itemize}
    \item $\mu$ is downward continuous: if $\{E_i\}_{i\geq1}$ is a decreasing sequence of sets such that $\mu(E_k)<\infty$ for some $k$, then $\mu(E_i)\to\mu(\cap_{i\geq1}E_i).$
    \item $\mu$ is $\sigma$-additive.
\end{itemize}
\end{prop}

\subsection{Integration}

\subsection{Measure and integration on product spaces}

\newpage
\section{Basic Probability}

\subsection{Basic notions}

\subsection{Independence and tail events}

\subsection{Convergence of random variables}

\subsection{Laws of large numbers}

\subsection{Tightness and characteristic functions}

\newpage
\section{Brownian Motion}

\newpage\appendix
\section{Proofs of Propositions}

\subsection{Measure theory}

\begin{prof}[1.1]
Suppose $\mu$ is downward continuous. Let $\{E_i\}_{i\geq1}$ be a collection of disjoint sets and $E=\amalg_{i\geq1}E_i$. Construct a decreasing sequence of sets $\{F_i\}_{i\geq1}$ by letting
$F_n=E\setminus\amalg_{i=1}^nE_i$ for all $n\geq1$, then by downward continuity
\[\mu(E)-\lim_{n\to\infty}\mu(\coprod_{i=1}^nE_i)=\lim_{n\to\infty}\mu(F_n)=\mu(\bigcap_{i\geq1}F_i)=0.\]
Then by finite additivity, $\mu(E)=\lim_{n\to\infty}\mu(\amalg_{i=1}^nE_i)=\lim_{n\to\infty}\sum_{i=1}^n\mu(E_i)=\sum_{i\geq1}\mu(E_i)$.\medbreak
On the other hand suppose $\mu$ is $\sigma$-additive.
Let $\{A_i\}_{i\geq1}$ be an increasing sequence of sets and $A=\cup_{i\geq1}A_i$. Construct a collection $\{B_i\}_{i\geq1}$ of disjoint sets by letting $B_1=A_1$ and $B_n=A_n\setminus A_{n-1}$ for all $n\geq2$, then by $\sigma$-additivity
\begin{equation}\label{(1)}
    \lim_{n\to\infty}\mu(A_n)=\lim_{n\to\infty}\mu(\coprod_{i=1}^nB_i)=\mu(\coprod_{i\geq1}B_i)=\mu(A).
\end{equation}
Now let $\{C_i\}_{i\geq1}$ be a decreasing sequence of sets and $C=\cap_{i\geq1}C_i$. Construct an increasing sequence of sets $\{D_i\}_{i\geq1}$ by letting $D_n=C_1\setminus C_{n+1}$ for all $n\geq1$, then by \hyperref[(1)]{(1)}
\[\mu(C_1)-\lim_{n\to\infty}\mu(C_n)=\lim_{n\to\infty}(D_n)=\mu(C_1\setminus C)=\mu(C_1)-\mu(C).\]
Hence $\mu(C)=\lim_{n\to\infty}\mu(C_n)$.
\end{prof}

\subsection{Basic probability}

\newpage\section{Definitions}

\subsection{Measure theory}

\begin{defin}
Class of subsets: let $\Omega$ be a set and $2^\Omega$ is power set.\newline
$\Lambda\subset2^\Omega$ is \textbf{$\lambda$-system} if\begin{itemize}
    \item $\Omega\in\Lambda$,
    \item $A\in\Lambda\Rightarrow A^c\in\Lambda$, and
    \item $\{A_i\in\Lambda\}_{i\geq1}\Rightarrow\amalg_{i\geq1}A_i\in\Lambda$.
\end{itemize}
$\Pi\subset2^\Omega$ is a \textbf{$\pi$-system} if\begin{itemize}
    \item $\varnothing\in\Pi$, and
    \item $A,B\in\Pi\Rightarrow A\cap B\in\Pi$.
\end{itemize}
$\mathcal{B}$ is a \textbf{Boolean algebra} if it is a $\pi$-system that satisfies $A\in\mathcal{B}\Rightarrow A^c\in\mathcal{B}$.\newline
$\mathcal{F}$ is a \textbf{$\sigma$-algebra} if it is a Boolean algebra that satisfies $\{A_i\in\mathcal{F}\}_{i\geq1}\Rightarrow\cup_{i\geq1}A_i\in\mathcal{F}$.\newline
\textbf{Borel $\sigma$-algebra} $\mathfrak{B}(\Omega)$ of $\Omega$ is the $\sigma$-algebra generated by open sets of $\Omega$.
\end{defin}

\subsection{Basic probability}

Suppose $d(\mu, \nu) = 0$. Then $\forall \, \epsilon > 0$, $\mu(A) \leq \nu(A^\epsilon) + \epsilon$ and $\nu(A) \leq \mu(A^\epsilon) + \epsilon$ for all $A \in \mathcal{B}(S)$. Suppose for contradiction that $\mu \neq \nu$. Then $\exists A \in \mathcal{B}(S)$ such that $\mu(A) \neq \nu(A)$. Suppose WLOG that $\mu(A) < \nu(A)$. Then $\exists \delta > 0$ such that $\mu(A) < \nu(A) - \delta$. Furthermore, since $\lim_{\epsilon \rightarrow 0} \mu(A^\epsilon) = \mu(A)$ by monotonicity \textcolor{blue}{(?)}, $\exists \, \epsilon > 0$ such that $\mu(A) < \mu(A^\epsilon) < \nu(A) - \delta$. Picking $\epsilon < \delta$, we obtain $\mu(A^\epsilon) < \nu(A) - \epsilon \; \Rightarrow \; \nu(A) > \mu(A^\epsilon) + \epsilon $. This is a contradiction, so we must have $\mu = \nu$. \\\\

Let $\varphi$ be a third probability measure on $S$. We will show that $d(\mu, \nu) \leq d(\mu, \varphi) + d(\varphi, \nu)$. We claim that if $\epsilon > d(\mu, \varphi)$ and $\epsilon' > d(\varphi, \nu)$, then $\epsilon + \epsilon' > d(\mu, \nu)$. Pick $\epsilon > d(\mu, \varphi)$ and $\epsilon' > d(\varphi, \nu)$ such that $\mu(A) \leq \varphi(A^\epsilon) + \epsilon$, $ \varphi(A) \leq \mu(A^\epsilon) + \epsilon$, $\varphi(A) \leq \nu(A^{\epsilon'}) + \epsilon'$, $ \text{ and } \nu(A) \leq \varphi(A^{\epsilon'}) + \epsilon' \text{ for all } A \in \mathcal{B}(S)$. We have: 
    \[\mu(A) - \varphi(A^\epsilon) + \varphi(A) - \nu(A^{\epsilon'}) \leq \epsilon + \epsilon'\]
    \[\varphi(A) - \mu(A^\epsilon) + \nu(A) - \varphi(A^{\epsilon'}) \leq \epsilon + \epsilon'\]

 \textcolor{blue}{Just need to show that $- \varphi(A^\epsilon) + \varphi(A)$ is arbitrarily small?}

\end{document}