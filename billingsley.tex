\documentclass[hidelinks,11pt]{article}

\usepackage[margin=1in]{geometry}
\usepackage{amsmath,amsthm,amsfonts}
\usepackage[utf8]{inputenc}
\usepackage{amssymb}
\usepackage[mathscr]{eucal}
\usepackage{graphicx}
\usepackage{listings}
\usepackage{xcolor}
\usepackage[OT1]{fontenc}
\usepackage{physics}
\usepackage{tikz-cd}
\usepackage{xpatch}
\usepackage{nicefrac}
\usepackage{mathtools}
\usepackage{environ}

\setlength\parindent{0pt}
\setcounter{section}{+1}

\newtheoremstyle{dotless}{}{}{\itshape}{}{\bfseries}{}{ }{}

\theoremstyle{definition}
\newtheorem*{defin}{DEF}
\newtheorem*{ex}{Exercise}

\theoremstyle{dotless}
\newtheorem*{corollary}{Corollary}
\newtheorem{prop}{PROP}[section]

\newtheoremstyle{named}{}{}{\upshape}{}{\bfseries}{}{.5em}{\thmname{#1} \thmnote{#3}}
\theoremstyle{named}
\newtheorem*{prof}{Proof}

\theoremstyle{remark}
\newtheorem*{remark}{Remark}

\usepackage{hyperref}
\hypersetup{colorlinks=false}

\DeclareMathOperator{\Var}{Var}
\DeclareMathOperator{\E}{\mathbb{E}}
\DeclareMathOperator{\R}{\mathbb{R}}
\DeclareMathOperator{\1}{\mathbf{1}}
\DeclareMathOperator{\p}{\mathbb{P}}



\begin{document}

Consider a metric space $(S,\rho)$ with topology induced by $\rho$, the Borel $\sigma$-algebra $\mathcal{S}$ of $S$, and a probability measure $\mathbb{P}$ on $(S,\mathcal{S})$.
\begin{defin}
$\p_n$ converges weakly to $\p$, or $\p_n\xrightarrow{w}\p$, if $\p_nf\to\p f$ for every bounded and continuous $f:S\to\R$.
\end{defin}
\begin{prop}\label{Prop 1.1}
Every probability measure $\p$ on $(S,\mathcal{S})$ is regular.
\end{prop}
\textcolor{magenta}{Since $F^\epsilon\searrow F$ for closed $F$, $\p$ is regular on closed sets. Since the $\sigma$-algebra generated by all closed sets is $\mathcal{S}$, we only need to show that the collection of sets which $\p$ is regular on is a $\sigma$-algebra.}\medbreak
This means that $\p$ is determined by the value $\p F$ for all closed $F$.
\begin{prop}
$\p$ is determined by the value $\p f$ for all bounded and uniformly continuous $f:S\to\mathbb{R}$.
\end{prop}
\textcolor{magenta}{For closed $F$ consider bounded and uniformly continuous $f(x)=(1-\epsilon^{-1}\rho( x,F))^+$ such that $\1_F(x)\leq f(x)\leq\1_{F^\epsilon}(x)$. By \hyperref[Prop 1.1]{Prop 1.1} we are done.}\medbreak
We usually call a subclass $\mathcal{S}\subset\mathcal{S}$ a separating class if $\p=\p'$ on $\mathcal{A}$ implies $\p=\p'$ on $\mathcal{S}$. By Dynkin's Theorem $\mathcal{A}$ is a separating class if it is a $\pi$-system that generates $\mathcal{S}$.
\begin{defin}
$\p$ is tight if for each $\epsilon>0$ there exists a compact $K\in\mathcal{S}$ such that $\p K>1-\epsilon$.
\end{defin}
\begin{prop}
If $S$ is separable and complete, then each probability measure $\p$ on $(S,\mathcal{S})$ is tight.
\end{prop}
\textcolor{magenta}{topological concept}

\begin{ex}
随机
\end{ex}



\end{document}