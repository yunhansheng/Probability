\documentclass[11pt]{article}

\usepackage{array,epsfig}
\usepackage{amsmath}
\usepackage{amsfonts}
\usepackage{amssymb}
\usepackage{amsxtra}
\usepackage{comment}
\usepackage{amsthm}
\usepackage{mathrsfs}
\usepackage{color}
\usepackage{physics}
\usepackage{nicefrac}

\usepackage{eucal}

\theoremstyle{definition}
\newtheorem{defn}{Definition}
\newtheorem{thm}{Theorem}
\newtheorem{cor}{Corollary}
\newtheorem*{remark}{Remark}
\newtheorem*{lem*}{Lemma}
\newtheorem{lem}{Lemma}
\newtheorem*{joke}{Joke}
\newtheorem{ex}{Example}
\newtheorem*{soln}{Solution}
\newtheorem{prop}{Proposition}

\setlength{\topmargin}{-.5 in}
\setlength{\oddsidemargin}{0in}
\setlength{\evensidemargin}{0in}
\setlength{\textheight}{9.in}
\setlength{\textwidth}{6.5in}
\pagestyle{empty}

\newcommand{\x}{\mathbf{x}}
\newcommand{\y}{\mathbf{y}}
\newcommand{\z}{\mathbf{z}}
\newcommand{\0}{\mathbf{0}}
\newcommand{\E}{\mathbb{E}}



\begin{document}

\begin{center}
{\Large\textbf STAT 38100 \hspace{0.1cm} Final}\\
\Large{Alex Sheng}
\end{center}

\vspace{0.2 cm}

\begin{enumerate}

\item Problem \#1\smallbreak
If $\sigma_n\to\sigma$, then $\sigma_n^2\to\sigma^2$. Since $X_n\sim\mathcal{N}(0,\sigma_n^2)$ is uniformly integrable, $X_n\xrightarrow{L^2}X$, which means that $X_n\xrightarrow{d}X$.\smallbreak
On the other hand, if $X_n\xrightarrow{d}X$, by Lévy Continuity Theorem $\phi_{X_n}(x)\to\phi_X(x)$. Since $X_n\sim\mathcal{N}(0,\sigma_n^2)$, $\phi_{X_n}(x)=e^{-\sigma_n^2x^2/2}$, from which we can see that $\phi_{X_n}(x)$ converges only when $\sigma_n^2\to\sigma^2$ for some $\sigma$, which implies $\sigma_n\to\sigma$. In this case, $\phi_X(x)=e^{-\sigma^2x^2/2}$ implies that $X\sim\mathcal{N}(0,\sigma^2)$.

\item Problem \#2\smallbreak



\item Problem \#3\smallbreak
When $n=2k$,
\begin{align*}
\frac{1}{n}P_n&=\frac{X_1X_2+X_2X_3+\cdots+X_{2k-1}X_{2k}}{2k}\\
&=\frac{X_1X_2+X_3X_4+\cdots+X_{2k-1}X_{2k}}{2k}+\frac{X_2X_3+X_4X_5+\cdots+X_{2k-1}X_{2k-2}}{2k}\\
&=A+B
\end{align*}
By independence $\E[X_iX_{i+1}]=\E[X_i]\E[X_{i+1}]=\mu^2$ for all $i\in\mathbb{N}$, hence we can apply Strong Law of Large Numbers and obtain $A,B\xrightarrow{a.s.}\mu^2$. Hence $\frac{1}{n}P_n\xrightarrow{a.s.}\mu^2$ for $n=2k$.\smallbreak
By the same token $\frac{1}{n}P_n\xrightarrow{a.s.}\mu^2$ for $n=2k-1$. Hence $\frac{1}{n}P_n\xrightarrow{a.s.}\mu^2$ for all $n\in\mathbb{N}$.

\item Problem \#4\smallbreak

\item Problem \#5\smallbreak
For any $K>0$, since $\mathbb{P}(B(t)>K\sqrt{t})=1-\phi(K)>0$ where $\phi(K)$ is the distribution function of $\mathcal{N}(0,1)$, we have $\mathbb{P}(\limsup_{n\to\infty}\frac{B(t)}{\sqrt{t}}=\infty)>0$. By Kolmogorov's 0-1 Law 
\[\mathbb{P}(\limsup_{n\to\infty}\frac{B(t)}{\sqrt{t}}=\infty)=1.\]
By the same token $\mathbb{P}(B(t)<-K\sqrt{t})=\phi(-K)>0$ implies
\[\mathbb{P}(\liminf_{n\to\infty}\frac{B(t)}{\sqrt{t}}=-\infty)=1.\]
Or alternatively use the reflection principle: $-\liminf_{n\to\infty}\frac{B(t)}{\sqrt{t}}=\limsup_{n\to\infty}\frac{-B(t)}{\sqrt{t}}=\infty$.

\item Problem \#6\smallbreak
(1) Let $X=\frac{1}{\sqrt{s}}B(S),Y=\frac{1}{\sqrt{t-s}}(B(t)-B(s))\sim\mathcal{N}(0,1)$, then $X$ and $Y$ are independent and
\begin{align*}
\mathbb{P}(B(t)>0\textrm{ and }B(s)>0)&=\mathbb{P}(X>0\textrm{ and }\sqrt{s}X+\sqrt{t-s}Y>0)\\
&=\int_{-\tan^{-1}\sqrt{\frac{s}{t-s}}}^\frac{\pi}{2}\,d\theta\int_0^\infty\frac{1}{2\pi}e^{-\frac{r^2}{2}}r\,dr\\
&=\frac{\frac{\pi}{2}+\tan^{-1}\sqrt{\frac{s}{t-s}}}{2\pi}
\end{align*}
(2) Since $B(1)\sim\mathcal{N}(0,1)$, $\E[B(1)^2]=$. Then
\begin{align*}
\E[B(1)^2B(2)B(3)]&=\E[B(1)^2B(2)(B(2)+B(3)-B(2))]\\
&=\E[B(1)^2B(2)^2]+\E[B(1)^2B(2)]\E[B(3)-B(2)]\\
&=\E[B(1)^2(B(1)+B(2)-B(1))^2]+0\\
&=\E[B(1)^4]+\E[B(1)^22B(1)]\E[B(2)-B(1)]+\E[B(1)^2]\E[(B(2)-B(1))^2]\\
&=3+0+1\cdot1=4
\end{align*}
by using the fact that $B(3)-B(2)$ and $B(2)-B(1)$ are independent standard normal variables.

\end{enumerate}

\end{document}