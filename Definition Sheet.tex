\documentclass[hidelinks,11pt]{article}

\usepackage[margin=1in]{geometry}
\usepackage{amsmath,amsthm,amsfonts}
\usepackage[utf8]{inputenc}
\usepackage{amssymb}
\usepackage[mathscr]{eucal}
\usepackage{graphicx}
\usepackage{listings}
\usepackage{xcolor}
\usepackage[OT1]{fontenc}
\usepackage{physics}
\usepackage{tikz-cd}
\usepackage{xpatch}
\usepackage{nicefrac}
\usepackage{mathtools}
\usepackage{environ}

\setlength\parindent{0pt}

\newtheoremstyle{dotless}{}{}{\itshape}{}{\bfseries}{}{ }{}

\theoremstyle{definition}
\newtheorem*{defin}{DEF}

\theoremstyle{dotless}
\newtheorem{prop}{PROP}[section]
\newtheorem*{corollary}{Corollary}

\theoremstyle{remark}
\newtheorem*{remark}{Remark}

\usepackage{hyperref}
\hypersetup{colorlinks=false}

\DeclareMathOperator{\Var}{Var}
\DeclareMathOperator{\E}{\mathbb{E}}
\DeclareMathOperator{\R}{\mathbb{R}}
\DeclareMathOperator{\1}{\mathbf{1}}



\begin{document}
\begin{center}
{\Large\textbf STAT 381-383-385 \hspace{0.1cm} Definition Page}\medbreak
\large{Alex Sheng}
\end{center}

\section{Measure Theory}

\subsection{Measures}
\begin{defin}
Class of subsets: let $\Omega$ be a set and $2^\Omega$ is power set.\newline
$\Lambda\subset2^\Omega$ is \textbf{$\lambda$-system} if\begin{itemize}
    \item $\Omega\in\Lambda$,
    \item $A\in\Lambda\Rightarrow A^c\in\Lambda$, and
    \item $\{A_i\in\Lambda\}_{i\geq1}\Rightarrow\amalg_{i\geq1}A_i\in\Lambda$.
\end{itemize}
$\Pi\subset2^\Omega$ is a \textbf{$\pi$-system} if\begin{itemize}
    \item $\varnothing\in\Pi$, and
    \item $A,B\in\Pi\Rightarrow A\cap B\in\Pi$.
\end{itemize}
$\mathcal{B}$ is a \textbf{Boolean algebra} if it is a $\pi$-system that satisfies $A\in\mathcal{B}\Rightarrow A^c\in\mathcal{B}$.\newline
$\mathcal{F}$ is a \textbf{$\sigma$-algebra} if it is a Boolean algebra that satisfies $\{A_i\in\mathcal{F}\}_{i\geq1}\Rightarrow\cup_{i\geq1}A_i\in\mathcal{F}$.\newline
\textbf{Borel $\sigma$-algebra} $\mathfrak{B}(\Omega)$ of $\Omega$ is the $\sigma$-algebra generated by open sets of $\Omega$.
\end{defin}

\begin{defin}
Measure function: let $\Omega$ be a set.\newline
$\mu:2^\Omega\to[0,\infty]$ is an \textbf{outer measure} if\begin{itemize}
    \item $\mu(\varnothing)=0$,
    \item $A\subset B\Rightarrow\mu(A)\leq\mu(B)$, and
    \item (countably subadditive) $\mu(\cup_{i\geq1}A_i)\leq\sum_{i\geq1}\mu(A_i)$.
\end{itemize}
$\mu:\mathcal{B}\to[0,\infty]$ is a \textbf{finitely-additive measure} if\begin{itemize}
    \item $\mu(\varnothing)=0$, and
    \item (finitely additive) $\mu(E\amalg F)=\mu(E)+\mu(F)$.
\end{itemize}
$\mu$ is a \textbf{pre-measure} if it is a finitely-additive measure that is \textbf{$\sigma$-additive}: $\mu(\amalg_{i\geq1}E_i)=\sum_{i\geq1}\mu(E_i)$.\newline
A \textbf{measure} is a pre-measure defined on a $\sigma$-algebra.
\end{defin}

\begin{defin}
Let $\mathcal{F}$ be a $\sigma$-algebra on $\Omega$. Duple $(\Omega,\mathcal{F})$ is a \textbf{measurable space}. Members of $\mathcal{F}$ are called \textbf{measurable sets}, members of $\mathfrak{B}(\Omega)$ are called \textbf{Borel sets} of $\Omega$. Let $\mu$ be a measure on $\mathcal{F}$, then triple $(\Omega,\mathcal{F},\mu)$ is a \textbf{measure space}.
\end{defin}

\begin{defin}
Let $\mu$ be an outer measure on $\Omega$, then $A\subset\Omega$ is \textbf{$\mu$-measurable} if for any $B\subset\Omega$,
\[\mu(B)=\mu(B\cap A)+\mu(B\cap A^c)\]
\end{defin}

\begin{defin}
A pre-measure $\mu$ on Boolean algebra $\mathcal{B}$ of $\Omega$ is \textbf{$\sigma$-finite} if there exists $\{A_i\in\mathcal{B}\}_{i\geq1}$ such that $\cup_{i\geq1}A_i=\Omega$ and $\mu(A_i)<\infty$ for all $i$.
\end{defin}

\begin{defin}
A function $f$ from a measurable space $(\Omega,\mathcal{F})$ to a topological space $(\Psi,\tau)$ is \textbf{measurable} if $f^{-1}(A)\in\mathcal{F}$ for every open set $A\in\tau$.
\end{defin}

\subsection{Integration}

\begin{defin}
simple function, integration (for simple function, measurable function, and measurable functions in $\overline{\mathbb{R}}$.
\end{defin}

\subsection{Measure and integration on product spaces}

\begin{defin}
Let $(X,\mathcal{X},\mu)$ and $(Y,\mathcal{Y},\nu)$ be measure spaces. The \textbf{product $\sigma$-algebra} $\mathcal{X}\otimes\mathcal{Y}$ is the $\sigma$-algebra generated by product sets $A\times B$ for $A\in\mathcal{X}$ and $B\in\mathcal{Y}$. The \textbf{product measure} $\mu\otimes\nu$ is the measure on $\mathcal{X}\otimes\mathcal{Y}$ such that $\mu\otimes\nu(A\times B)=\mu(A)\nu(B)$ for all $A\in\mathcal{X}$ and $B\in\mathcal{Y}$.
\end{defin}

\section{Basic Probability}

\subsection{Basic notions}

\subsection{Independence and tail events}

\begin{defin}
Independence: let $(\Omega,\mathcal{F},\mathbb{P})$ be the probability space.\begin{itemize}
    \item Two events $A,B\in\mathbb{F}$ are \textbf{independent} if $\mathbb{P}(A\cap B)=\mathbb{P}(A)\mathbb{P}(B)$.
    \item Two collections $\mathcal{H},\mathcal{G}\subset\mathcal{F}$ are \textbf{independent} if $\mathbb{P}(G\cap H)=\mathbb{P}(G)\mathbb{P}(H)$ for all $G\in\mathcal{G}$ and $H\in\mathcal{H}$.
    \item Two random variables $X$ and $Y$ are \textbf{independent} if $\sigma(X)$ and $\sigma(Y)$ are independent as $\sigma$-algebras, where $\sigma(f)$ is the collection of all pre-images of Borel sets under $f$.
    \item An arbitrary collection of events/$\sigma$-algebras/random variables is \textbf{independent} if any finite subcollection is independent.
\end{itemize}
\end{defin}

\begin{defin}
Let $\{X_i\}_{i\geq1}$ be a sequence of random variables defined on measure space $(\Omega,\mathcal{F},\mathbb{P})$. The \textbf{tail $\sigma$-algebra} of $\{X_i\}$ is
\[\mathcal{T}(X_1,X_2,\cdots)=\bigcap_{i\geq1}\sigma(X_i,X_{i+1},\cdots)=\bigcap_{i\geq1}\sigma(\{\omega\in\Omega:X_i(\omega)\leq t\textrm{ for all }t\in\mathbb{R}\})\]
\end{defin}

\subsection{Convergence of random variables}

\begin{defin}
A \textbf{Polish space} is a completely metrizable separable metric space.
\end{defin}

\begin{defin}Convergence: let $\{X_n\}_{n\geq1}$ be a sequence of random variables in a measure space $(\Omega,\mathcal{F},\mathbb{P})$ taking values in a Polish space $(\mathcal{S},\rho)$.\begin{itemize}
    \item $\{X_n\}$ \textbf{converges almost surely} to $X$ ($X_n\xrightarrow{a.s.}X$) if $X_n\to X$ alsmot surely.
    \item $\{X_n\}$ \textbf{converges in probability} to $X$ ($X_n\xrightarrow{\mathbb{P}}X$) if $\mathbb{P}(\rho(X_n,X)\geq\epsilon)\to0$ for any $\epsilon>0$.
    \item $\{X_n\}$ \textbf{converges in distribution/law} to $X$ ($X_n\xrightarrow{d}X$) if $F_{X_n}\to F_X$ on points where $F_x$ is continuous, where $F_{X_n}$ and $F_X$ are distribution functions of $X_n$ and $X$ respectively.
    \item $\{X_n\}$ \textbf{converges in $L^p$} to $X$ ($X_n\xrightarrow{L^p}X$) if $\norm{X_n-X}_p\to0$.
    \item $\{X_n\}$ \textbf{converges weakly} to $X$ ($X_n\xrightarrow{weak}X$) if $\E[f(X_n)]\to\E(f(X))$ for all bounded and continuous $f:\mathcal{S}\to\mathbb{R}$.
\end{itemize}
Let $\{\mu_n\}_{n\geq1}$ be a sequence of probability measures, then $\mu_n$ \textbf{converges weakly} to $\mu$ ($\mu_n\xrightarrow{weak}\mu$) if for any bounded and continuous function $f:\mathcal{S}\to\mathbb{R}$:
\[\int_\mathcal{S}f\,d\mu_n\to\int_\mathcal{S}f\,d\mu\]
\end{defin}

\subsection{Laws of large numbers}

\subsection{Tightness and characteristic functions}

\begin{defin}Tightness:\begin{itemize}
    \item A sequence of random variables $\{X_n\}_{n\geq1}$ is a \textbf{tight family} if for any $\epsilon>0$ there exists some $K$ such that $\sup_n\mathbb{P}(\abs{X_n}\geq K)\leq\epsilon$.
    \item A sequence of probability measures $\{\mu_n\}_{n\geq1}$ on Polish space $(\mathcal{S},\rho)$ is \textbf{tight} if for any $\epsilon>0$ there is a compact $K\subset\mathcal{S}$ such that $\mu_n(K)\geq1-\epsilon$ for all $n$.
\end{itemize}
\end{defin}

\section{Construction of Brownian Motion}

\begin{defin}
For each $n\in\mathbb{N}$ and $t_i$ for $i\in[1,n]$, the \textbf{finite-dimensional projection map} on $C[0,1]$ is
\begin{align*}
    \pi_{t_i}:C[0,1]&\to\mathbb{R}^n\\
    f&\mapsto(f(t_1),f(t_2),\cdots,f(t_n))
\end{align*}
Let $\mu$ be a probability measure on $C[0,1]$, the pushforward of $\mu$ under a projection map is a \textbf{finite-dimensional distribution}.
\[\begin{tikzcd}
{C[0,1]} \arrow[d, "\mu"'] \arrow[r, "\pi_{t_i}"] & \mathbb{R}^n \arrow[ld, "\pi_{t_i\ast}\mu=\mu\circ\pi_{t_i}^{-1}"] \\
{[0,1]}                                           &                                                                   
\end{tikzcd}\]
One can change $C[0,1]$ to $C[0,\infty)$.
\end{defin}

\begin{defin}
Let $\{X_n\}_{n\geq1}$ be a sequence of independent and identically distributed random variables with mean $0$ and variance $1$. Let $\{B_n\}_{n\geq1}$ be a sequence of $C[0,1]$-valued random variables such that $B_n(0)=0$, $B_n(t)=\frac{1}{\sqrt{n}}\sum_{j=1}^iX_j$ for $t=\frac{i}{n}$ and $1\leq i\leq n$, and $B_n$ between $\frac{i-1}{n}$ and $\frac{i}{n}$ are obtained by linear interpolation. The weak limit $B$ of $B_n$ is called the \textbf{Brownian motion} on $C[0,1]$ (on the time interval $[0,1]$), and its law \textbf{Wiener measure}.\medbreak
Let $\{B^n\}_{n\geq1}$ be a sequence of independent and identically distributed Brownian motions on $C[0,1]$, then for $k\in\mathbb{N}$ and $k\leq t\leq k+1$ define \textbf{standard Brownian motion}:
\[B(t)=\sum_{j=1}^kB^j(t)+B^k(t-k).\]
\end{defin}



\end{document}