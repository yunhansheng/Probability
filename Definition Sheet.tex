\documentclass[hidelinks,11pt]{article}

\usepackage[margin=1in]{geometry}
\usepackage{amsmath,amsthm,amsfonts}
\usepackage[utf8]{inputenc}
\usepackage{amssymb}
\usepackage[mathscr]{eucal}
\usepackage{graphicx}
\usepackage{listings}
\usepackage{xcolor}
\usepackage[OT1]{fontenc}
\usepackage{physics}
\usepackage{tikz-cd}
\usepackage{xpatch}
\usepackage{nicefrac}
\usepackage{mathtools}
\usepackage{environ}

\setlength\parindent{0pt}

\newtheoremstyle{dotless}{}{}{\itshape}{}{\bfseries}{}{ }{}

\theoremstyle{definition}
\newtheorem*{defin}{DEF}

\theoremstyle{dotless}
\newtheorem{prop}{PROP}[section]
\newtheorem*{corollary}{Corollary}

\theoremstyle{remark}
\newtheorem*{remark}{Remark}

\usepackage{hyperref}
\hypersetup{colorlinks=false}

\DeclareMathOperator{\Var}{Var}
\DeclareMathOperator{\E}{\mathbb{E}}
\DeclareMathOperator{\R}{\mathbb{R}}
\DeclareMathOperator{\1}{\mathbf{1}}



\begin{document}
\begin{center}
{\Large\textbf STAT 381-383-385 \hspace{0.1cm} Definition Page}\medbreak
\large{Alex Sheng}
\end{center}

\section{Measure Theory}

\section{Basic Probability}

\subsection{Basic notions}

\subsection{Independence and tail events}

\begin{defin}
Independence: let $(\Omega,\mathcal{F},\mathbb{P})$ be the probability space.\begin{itemize}
    \item Two events $A,B\in\mathbb{F}$ are \textbf{independent} if $\mathbb{P}(A\cap B)=\mathbb{P}(A)\mathbb{P}(B)$.
    \item Two collections $\mathcal{H},\mathcal{G}\subset\mathcal{F}$ are \textbf{independent} if $\mathbb{P}(G\cap H)=\mathbb{P}(G)\mathbb{P}(H)$ for all $G\in\mathcal{G}$ and $H\in\mathcal{H}$.
    \item Two random variables $X$ and $Y$ are \textbf{independent} if $\sigma(X)$ and $\sigma(Y)$ are independent as $\sigma$-algebras, where $\sigma(f)$ is the collection of all pre-images of Borel sets under $f$.
    \item An arbitrary collection of events/$\sigma$-algebras/random variables is \textbf{independent} if any finite subcollection is independent.
\end{itemize}
\end{defin}

\begin{defin}
Let $\{X_i\}_{i\geq1}$ be a sequence of random variables defined on measure space $(\Omega,\mathcal{F},\mathbb{P})$. The \textbf{tail $\sigma$-algebra} of $\{X_i\}$ is
\[\mathcal{T}(X_1,X_2,\cdots)=\bigcap_{i\geq1}\sigma(X_i,X_{i+1},\cdots)=\bigcap_{i\geq1}\sigma(\{\omega\in\Omega:X_i(\omega)\leq t\textrm{ for all }t\in\mathbb{R}\})\]
\end{defin}

\subsection{Convergence of random variables}

\begin{defin}
A \textbf{Polish space} is a completely metrizable separable metric space.
\end{defin}

\begin{defin}Convergence: Let $\{X_n\}_{n\geq1}$ be a sequence of random variables in a measure space $(\Omega,\mathcal{F},\mathbb{P})$ taking values in a Polish space $(\mathcal{S},\rho)$.\begin{itemize}
    \item $\{X_n\}$ \textbf{converges almost surely} to $X$ ($X_n\xrightarrow{a.s.}X$) if $X_n\to X$ alsmot surely.
    \item $\{X_n\}$ \textbf{converges in probability} to $X$ ($X_n\xrightarrow{\mathbb{P}}X$) if $\mathbb{P}(\rho(X_n,X)\geq\epsilon)\to0$ for any $\epsilon>0$.
    \item $\{X_n\}$ \textbf{converges in distribution} to $X$ ($X_n\xrightarrow{d}X$) if $F_{X_n}\to F_X$ on points where $F_x$ is continuous, where $F_{X_n}$ and $F_X$ are distribution functions of $X_n$ and $X$ respectively.
    \item $\{X_n\}$ \textbf{converges in $L^p$} to $X$ ($X_n\xrightarrow{L^p}X$) if $\norm{X_n-X}_p\to0$.
    \item $\{X_n\}$ \textbf{converges weakly} to $X$ ($X_n\xrightarrow{weak}X$) if $\E[f(X_n)]\to\E(f(X))$ for all bounded and continuous $f:\mathcal{S}\to\mathbb{R}$.
\end{itemize}
Let $\{\mu_n\}_{n\geq1}$ be a sequence of probability measures, then $\mu_n$ \textbf{converges weakly} to $\mu$ ($\mu_n\xrightarrow{weak}\mu$) if for any bounded and continuous function $f:\mathcal{S}\to\mathbb{R}$:
\[\int_\mathcal{S}f\,d\mu_n\to\int_\mathcal{S}f\,d\mu\]
\end{defin}

\subsection{Tightness and characteristic functions}

\begin{defin}
A sequence of random variables $\{X_n\}$ is a \textbf{tight family} if for any $\epsilon>0$ there exists some $K$ such that $\sup_n\mathbb{P}(\abs{X_n}\geq K)\leq\epsilon$.
\end{defin}



\end{document}